\documentclass[11pt]{article}
\begin{document}

% This file contains markup for the BSSM4GSQ user manual.

\title{BSSM4GSQ code package}
\author{Michael E Sparks}
\date{\today}
\maketitle

\section{Getting Started}
This software facilitates generation of splice site probabilities for
use with \texttt{GeneSeqer}, and training data with which to
build splice site parameters using the \texttt{gthbssmbuild}
tool in the \texttt{GenomeThreader} package.
\texttt{BSSM4GSQ} can process either plain text \texttt{GeneSeqer}
or gthXML v1.0 (or later) output.
gthXML output can be produced natively by the \texttt{gth}
program of \texttt{GenomeThreader}, and can be produced from plain text
\texttt{GeneSeqer} output using the \texttt{GSQ2XML.pl} script,
available from either \break
\texttt{http://www.genomethreader.org} or \break
\texttt{http://www.public.iastate.edu/$\sim$mespar1/gthxml/}.
The end user may wish to study the following reports prior to
using this software:
\begin{enumerate} 
  \item
  Brendel V, Xing L, Zhu W. (2004) Gene structure prediction from consensus
  spliced alignment of multiple ESTs matching the same genomic locus.
  \emph{Bioinformatics}. \textbf{20}:1157-69.
  \item
  Sparks ME and Brendel V. (2005) Incorporation of splice site probability models
  for non-canonical introns improves gene structure prediction in plants.
  \emph{Bioinformatics}. \textbf{21}:iii20-iii30.
\end{enumerate}

\section{Directions}

\begin{enumerate}
  \item
  Verify that the following executables are present in the bin/ directory.

  \begin{enumerate}
    \item
    \texttt{indexFasSeq}
    \item
    \texttt{BSSM\_build}
    \item
    \texttt{BSSM\_print}
  \end{enumerate}

  If any of these are absent, cd to the src/ directory and issue ``make".
  (This step is, however, optional, as the \texttt{Mktraindata.sh} and
  \texttt{Mkbssmparm.sh} scripts will build the files, when necessary.)

  \item
  There are two subdirectories in the
  input directory, gsq/ and fas/.  You absolutely must meet the
  following requirements to make the code work.

  \begin{enumerate}
    \item
    There must be a one-to-one correspondence between files in
    these two directories.
    \item
    Files in the fas/ directory must have an extension of ``.fas"
    and those in the gsq/ directory must have one of either ``.gsq"
    or ``.xml", for plain text \texttt{GeneSeqer} or gthXML formatted
    data files, respectively.  You cannot mix plain text and gthXML
    input files.
    \item
    The basename prefixes of cognate fas/gsq file pairs, i.e., the
    substrings prior to ``.gsq" or ``.fas", must be identical.
    \item
    All references made to a genomic template in the spliced alignment
    output file must be identical to the file's ``file handle"
    mentioned above.  This essentially mandates that each fas/gsq
    cognate file pair correspond to one genomic sequence and its
    spliced alignment annotation, respectively.
  \end{enumerate}

  There are sample training data in the input/sample/ directory.  These
  data will not generate any meaningful probabilities, and are
  only intended to demo the system.
   
  \item
  Edit \texttt{Mktraindata.sh} such that the FORMAT variable is
  set correctly for the files placed in the input/gsq/ directory; this is
  described explicitly in the header of the script.
  Run \texttt{Mktraindata.sh}.  This produces exon and intron data, sorted
  according to phase and placed in the output/exons\_introns/ directory,
  and sampled, phase-sorted BSSM training data placed in the
  output/training\_data/ directory.  In each of these directories, data
  will be written to a subdirectory named according to the
  donor/acceptor dinucleotide termini trained for.  (If this is
  unclear, inspect the contents of the output directories after
  unpacking this code, run the script, and look at them again.)

  \texttt{Mktraindata.sh} processes GT-AG introns by default.
  For other types, tune the DON and ACC variables (set these in CAPITAL letters!)
  and run it again.  This will not overwrite any existing output in
  the training\_data/ or exons\_introns/ directories so long as a
  different DON/ACC combination is used.  Rerunning the script
  using a DON/ACC pair whose results were already recorded will
  cause the original data to be overwritten.

  \item
  Run \texttt{Mkbssmparm.sh}.  The script will solicit some configuration
  information:

  \begin{enumerate}
    \item
    Name of output file (``foo.bssm")
    \item
    Root directory of training data.  (If you haven't done
    anything non-standard up to this point, it should be safe
    to just say ``y" here.)
    \item
    Build GT model?  For GT-AG parameterizations.  (If you trained
    for these intron types, responding with ``y" will put the
    probabilities in your *.bssm file.  Else, say ``n".)
    \item
    Build GC model?  For GC-AG parameterizations.  (If you trained
    for these intron types, responding with ``y" will put the
    probabilities in your *.bssm file.  Else, say ``n".)
    \item
    File to write ascii data to (``foo.bssm.ascii")
  \end{enumerate}

  (Users of the \texttt{GenomeThreader} package
  (\texttt{http://www.genomethreader.org})
  should note that these binary *.bssm files are \textbf{not compatible} with
  those used by that system.  Code in \texttt{BSSM4GSQ} implementing the
  weight array matrix development routines
  was adapted to implement the \texttt{gthbssmbuild} program
  of \texttt{GenomeThreader}; given input data like that produced in steps
  1-3 above, \texttt{gthbssmbuild} will generate \texttt{GenomeThreader}-compatible
  binary *.bssm files.
  Please refer to the \texttt{GenomeThreader} manual or
  contact Gordon Gremme at \texttt{gremme@zbh.uni-hamburg.de} for details.)

  The \texttt{BSSM\_print} utility allows the user to generate an ascii
  representation of the trained splice site probability matrices.
  The *.bssm.ascii file presents the weight array matrices in the following
  order:

  \begin{verbatim}
     for TERMINAL in (0-1):
       for HYPOTHESIS in (0-6):
         print transition probabilities
  \end{verbatim}

  where for TERMINAL, 0 and 1 index donor and acceptor sites,
  respectively; and for HYPOTHESIS, 0, 1, 2, 3, 4, 5, 6 index the T1,
  T2, T0, F1, F2, F0 and Fi hypotheses, respectively.

  The user will, unfortunately, have to manually splice the new probabilities
  into the daPbm7.* header files included with the \texttt{GeneSeqer} source
  distribution (adjust the GU\_7/GC\_7 and AG\_7 arrays, the name\_model
  array, and the NMDLS macro accordingly) to incorporate the models
  into the software.  There is a script provided in the src/plscripts/ directory,
  \texttt{punctuation.pl}, that will assist in adding syntactical markup
  to allow copying/pasting results into the header files, e.g.,

  \begin{verbatim}
$ cat something.bssm.ascii | ./punctuation.pl
  \end{verbatim}
  You can verify that your parameter file contains valid probability
  mass distributions by using the \texttt{verify\_pmf.pl} script
  in the src/plscripts/ directory, e.g.,

  \begin{verbatim}
$ cat something.bssm.ascii | ./verify_pmf.pl
  \end{verbatim}
  Please see the commentary in that file for more details;
  output of a row of zeros is expected, and does not signal
  an error.
\end{enumerate}

\section{Contact Info}
If you have questions, concerns, etc.,
please email me at \texttt{mespar1@iastate.edu}.

\end{document}
